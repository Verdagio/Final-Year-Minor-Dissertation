\chapter{Conclusion}
About three pages.

\begin{itemize}
\item Briefly summarise your context and ob-jectives (a few lines).
\item Highlight your findings from the evaluation section / chapter and any opportuni-ties identified.
\end{itemize}

% Maybe ???
% Thoughts and Observations ----------------------------
\section{Thoughts and Observations}
Testing makes for robust and durable code \\
Testing highlights and identifies bugs \\
Using Docker and other technologies requires good unix/linux command line skills and helps improve them \\
Writing asynchronous code in Node.js is made much cleaner and readable using async/await vs callbacks or promises
Writing high quality code does not require a user interface or client application for the means of demonstration or testing


% MOVE TO CONCLUSION
% Conclusion -------------------------------------------
\section{Microservice Architecture}
While Microservice Architecture is a fantastic solution for many reoccurring problems in an application design, it should be considered by no means a silver a bullet \cite{BuildingMicroServices}. Microservice Architecture certainly adds a number of complexities and is often considered very risky to begin with as adopting this style can be very difficult to get right. Every application has different requirements and perhaps a simpler approach should be taken in the beginning. It is not uncommon to start with a monolith style architecture and later refactor to a Microservice based design. CTO of NPM Laurie Voss explains in his keynote how NPM began as a monolith and later refactored to a Microservice Architecture and is now the world's largest software registry\cite{npm}. So while Microservices and indeed Docker may not be the cure to address every aspect of software design and architecture, they both definitely have a lot promising and helpful tools to offer.

% Docker
\section{Docker as a continuous development tool}
Docker as a development tool is relatively easy to pick up and implement, due to the fact that it is well supported and documented by its community. However, it comes with a steep learning curve as you traverse deeper into its more advanced features and functionality. The isolated environments that Docker containers provide, enables work across different machines regardless of the underlying infrastructure, which in the case of this project, provided an easy work flow during cross examination, merging, and service integration. Going forward Docker will continue to make this project more and more versatile, secure, quick, and easy to maintain because of the Docker ethos. Long after the project has been deployed and it is in use by the product owner, integration engineers and developer operations teams should find processes like: onboarding new services, or analyzing and repairing problematic services easy to identify, then address, seamlessly and fast. Finally with the exponential growth in adoption rate of Docker over the last couple of years, virtual machines may very well be a thing of the past in years to come.

% Future Development -------------------------------------
\section{Future Development}


